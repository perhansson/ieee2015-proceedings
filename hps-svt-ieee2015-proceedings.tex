%% NSS-MIC_Instructions.tex
%% 8/2007
%% By Bo Yu (yu@bnl.gov)
%% based on:
%% bare_jrnl.tex
%% V1.3
%% 2007/01/11
%% by Michael Shell
%% see http://www.michaelshell.org/
%% for current contact information.
%%
%% This is a skeleton file demonstrating the use of IEEEtran.cls
%% (requires IEEEtran.cls version 1.7 or later) with an IEEE journal paper.
%%
%% Support sites:
%% http://www.michaelshell.org/tex/ieeetran/
%% http://www.ctan.org/tex-archive/macros/latex/contrib/IEEEtran/
%% and
%% http://www.ieee.org/


%%*************************************************************************
%% Legal Notice:
%% This code is offered as-is without any warranty either expressed or
%% implied; without even the implied warranty of MERCHANTABILITY or
%% FITNESS FOR A PARTICULAR PURPOSE! 
%% User assumes all risk.
%% In no event shall IEEE or any contributor to this code be liable for
%% any damages or losses, including, but not limited to, incidental,
%% consequential, or any other damages, resulting from the use or misuse
%% of any information contained here.
%%
%% All comments are the opinions of their respective authors and are not
%% necessarily endorsed by the IEEE.
%%
%% This work is distributed under the LaTeX Project Public License (LPPL)
%% ( http://www.latex-project.org/ ) version 1.3, and may be freely used,
%% distributed and modified. A copy of the LPPL, version 1.3, is included
%% in the base LaTeX documentation of all distributions of LaTeX released
%% 2003/12/01 or later.
%% Retain all contribution notices and credits.
%% ** Modified files should be clearly indicated as such, including  **
%% ** renaming them and changing author support contact information. **
%%
%% File list of work: IEEEtran.cls, IEEEtran_HOWTO.pdf, bare_adv.tex,
%%                    bare_conf.tex, bare_jrnl.tex, bare_jrnl_compsoc.tex
%%*************************************************************************
\documentclass[journal]{IEEEtran}
\usepackage{graphicx}

\newcommand{\Aprime}{\ensuremath{\mathrm{A}^\prime}}


\begin{document}
\title{The Silicon Vertex Tracker for the \\Heavy Photon Search Experiment}
%
% author names and IEEE memberships
% note positions of commas and nonbreaking spaces ( ~ ) LaTeX will not break
% a structure at a ~ so this keeps an author's name from being broken across
% two lines.
% use \thanks{} to gain access to the first footnote area
% a separate \thanks must be used for each paragraph as LaTeX2e's \thanks
% was not built to handle multiple paragraphs
%

%\author{P..~Hansson~Adrian,~\IEEEmembership{Member,~IEEE,}
%        Second~B.~Author,~\IEEEmembership{Fellow,~OSA,}
%        and~Third~C.Author,~\IEEEmembership{Life~Fellow,~IEEE}% <-this % stops a space
%\thanks{Manuscript received November 4, 2007. (Write the date on which you submitted your paper for review.) This work was supported in part by the U.S. Department of Commerce under Grant No. BS123456 (sponsor acknowledgment goes here).}% <-this % stops a space
%\thanks{Full names of authors are preferred in the author field, but are not required. Put a space between authors' initials. Do not use all uppercase for authors' surnames.}%
%\thanks{F. A. Author is with the National Institute of Standards and Technology, Boulder, CO 80303 USA (telephone: 303-497-3650, e-mail: author @boulder.nist.gov).}%
%\thanks{S. B. Author, Jr., was with Rice University, Houston, TX 77005 USA. He is now with the Department of Physics, Colorado State University, Ft. Collins, CO 80523 USA (telephone: 970-491-6206, e-mail: author@lamar. colostate.edu).}%
%\thanks{T. C. Author is with the Electrical Engineering Department, University of Colorado, Boulder, CO 80309 USA, on leave from the National Research Institute for Metals, Tsukuba, Japan (e-mail: author@nrim.go.jp).}%
%}

\author{Per~Hansson~Adrian$^1$, on behalf of the HPS Collaboration\\
$^1$SLAC National Accelerator Laboratory, Menlo Park, CA, USA

\thanks{Manuscript received November 23, 2015. Work supported by 
the U.S. Department of Energy under contract number DE-AC02-76SF00515, 
the National Science Foundation,  
French Centre National de la Recherche Scientifique and 
Italian Istituto Nazionale di Fisica Nucleare. Authored by Jefferson Science 
Associates, LLC under under U.S. Department of Energy contract No. DE-AC05-06OR23177.}%
%\thanks{Rouven Essig is supported in part by the Department of Energy Early Career research program 
%DESC0008061and by a Sloan Foundation Research Fellowship. Authored by Jefferson Science 
%Associates, LLC under under U.S. Department of Energy contract No. DE-AC05-06OR23177.
}

\maketitle
\pagestyle{empty}
\thispagestyle{empty}

\begin{abstract}
The Heavy Photon Search (HPS) is a new, dedicated experiment at Thomas Jefferson National Accelerator Facility (JLab) to search for a massive vector boson, the heavy photon (a.k.a. dark photon, \Aprime{}), in the mass range 20-500~MeV/c$^{2}$ and with a weak coupling to ordinary matter. An \Aprime{} can be radiated from an incoming electron as it interacts with a charged nucleus in the target, accessing a large open parameter space where the \Aprime{} is relatively long-lived, leading to displaced vertices. HPS searches for these displaced \Aprime{} to e$^+$e$^-$ decays using actively cooled silicon microstrip sensors with fast readout electronics placed immediately downstream of the target and inside a dipole magnet to instrument a large acceptance with a relatively small detector. With typical particle momenta of 0.5-2~GeV/c, the low material budget of 0.7\% $X_0$ per tracking layer is key to limiting the dominating multiple scattering uncertainty and allowing efficient separation of the decay vertex from the prompt QED trident background processes. Achieving the desired low-mass acceptance requires placing the edge of the silicon only 0.5~mm from the electron beam. This results in localized hit rates above 4~MHz/mm$^2$ and radiation levels above $10^{14}$ 1~MeV neutron equivalent dose. Hit timing at the ns level is crucial to reject out-of time hits not associated with the \Aprime{} decay products from the almost continuous CEBAF accelerator beam. To avoid excessive beam-gas interactions the tracking detector is placed inside the accelerator beam vacuum envelope and is retractable to allow safe operation in case of beam motion. This contribution will discuss the design, construction 
and first results from the first data-taking period in the spring of 2015. 
\end{abstract}

%\begin{IEEEkeywords}
%IEEEtran, journal, \LaTeX, paper, template.
%\end{IEEEkeywords}


\section{Introduction}
% The very first letter is a 2 line initial drop letter followed
% by the rest of the first word in caps.
% 
% form to use if the first word consists of a single letter:
% \IEEEPARstart{A}{demo} file is ....
% 
% form to use if you need the single drop letter followed by
% normal text (unknown if ever used by IEEE):
% \IEEEPARstart{A}{}demo file is ....
% 
% Some journals put the first two words in caps:
% \IEEEPARstart{T}{his demo} file is ....
% 
% Here we have the typical use of a "T" for an initial drop letter
% and "HIS" in caps to complete the first word.

\IEEEPARstart{R}{ecent} astrophysical results ~\cite{pamela,fermi} have generated intense interest in physics models 
beyond the Standard Model with a new force, mediated by a massive, sub-GeV scale, 
U(1) gauge boson (a.k.a. the Heavy Photon, Dark Photon or \Aprime{}) that couples very weakly to 
ordinary matter through "kinetic mixing"~\cite{nima,holdom}. The existence of such a new force is in 
accord with astrophysical and cosmological constraints. Its weak coupling to the electric charge could be 
the only non-gravitational window into the existence of hidden sectors consisting of particles that do not 
couple to any of the known forces that are common in many new physics scenarios~\cite{Hewett:2012ns}.

The Heavy Photon Search experiment (HPS) is a new fixed-target experiment~\cite{proposal_full}
specifically designed to discover an \Aprime{} with m$_{\Aprime}=20-500$~MeV, produced through bremsstrahlung 
in a tungsten target and decaying into an $e^{+}e^{-}$ pair. 
In particular, the HPS experiment has sensitivity to the challenging region with small cross sections out of 
reach from collider experiments and where thick absorbers, as used in beam-dump experiments to 
reject backgrounds, are ineffective due to the relatively short \Aprime{} decay length ($<1$~m).  
This is accomplished 
by placing a compact silicon tracking and vertex detector (SVT) in a magnetic field, immediately downstream (10~cm) 
of a thin ($\sim 0.125\%~X_{0} $) target to reconstruct the mass and decay vertex position of  the \Aprime{}.


HPS  runs in Hall~B at Thomas Jefferson National Accelerator Facility (JLab) using the CEBAF 
accelerator electron beam with an energy of 1.05~GeV and 50~nA current, with planned operation of up to 6.6~GeV and 450~nA. 
The kinematics of \Aprime{} production 
typically results in final state particles within a few degrees of the incoming beam, especially at low $m_{\Aprime}$ . 
Because of this, the apparatus must accommodate passage of the beam 
downstream of the target and operate as close to the beam as possible. Because background rates in this region from the 
scattered beam are very large, a fast lead-tungstate crystal calorimeter trigger with 250~MHz FADC readout [ref] and 
excellent time tagging of hits is used to select interesting data and reduce the bandwidth required to transfer data from the 
detector.  This method of background reduction is the motivation for operating HPS in a
nearly continuous beam: in a beam with large per-bunch charge, background from a single bunch would fully occupy the detector 
at the required beam intensity. 





\section{The Silicon Vertex Tracking Detector}

\begin{figure}[]
\centering
\includegraphics[width=0.4\textwidth]{occupancy.pdf}
\caption{Occupancy per strip ($60~\mu$m readout pitch) in Layer 1 of the SVT for 
8~ns of beam at 400~nA.}
\label{fig:occupancy}
\end{figure}
\begin{figure}[]
\centering
\includegraphics[width=0.4\textwidth]{PastedGraphic-1.png}
\caption{View of the SVT from upstream before installation of the target and final cabling.}
\label{fig:reach}
\end{figure}
At beam energies necessary to achieve sensitivity to \Aprime{} in the most interesting mass range for HPS, 
multiple scattering dominates the measurement uncertainty, and in particular dictates the 
achievable vertex position resolution for any practical material budget. The main design guidelines are 
therefore to minimize the material budget in the tracking volume, and the distance to the beam in order 
to increase acceptance for low $m_{\Aprime}$, while keeping the occupancy under control.   
Furthermore, the whole tracker has to operate in vacuum to avoid secondary backgrounds from 
beam gas interactions, and have retractable tracking planes and easy access for sensor replacement to 
increase safety. 
Given the high hit density, the fast time response, and good resolution and radiation hardness needed; silicon microstrip 
sensors are the technology of choice for the tracker. Pixel sensors suitable for instrumenting our large acceptance are either too slow
or have an unacceptable material budget.
Each of the six tracking layers, arranged in two halves both above and below the beam, consists of placing Hamamatsu Photonics Corporation silicon microstrip sensors back-to-back, with 
50 or 100~mrad stereo angle.  These are $320~\mu$m thick, $p$+-on-$n$, 
AC coupled, polysilicon-biased sensors with 60 (30)~$\mu$m readout (sense) pitch, readily available at low cost from the cancelled 
D0 RunIIb upgrade~\cite{d0run2b}. The overall area of 1440~cm$^2$ has in total 23,040 channels. 
The optimized design, with the first layer placed only 10~cm downstream of the target to give 
excellent 3D vertexing performance, has a 15~mrad dead zone above and below the beam axis, putting the active silicon only 
1.5~mm from the the center of the beam where hit densities reach 4~MHz/cm$^2$ with per-strip 
occupancies kept $<1$\%, as shown in Fig.~\ref{fig:occupancy}. 
\begin{table}
\centering
\begin{tabular}{|lcc|}
\hline
Layer $\rightarrow$& 1-3 & 4-6 \\ 
\hline
$z$ pos. (cm)  & 10-30 & 50-90  \\
Stereo angle  & $90^{\circ}$ & 50~mrad  \\
Bend res. ($\mu$m)  & $\approx 6$ & $\approx6$  \\
Stereo res. ($\mu$m)  & $\approx 6$ & $\approx130$  \\
\hline
\end{tabular}
\caption{Main tracker parameters.}
\label{tab:svtparams}
\end{table}
To resolve overlapping hits in time and thus help to reject background and improve pattern recognition 
in the area closest to the beam, a 2~ns single hit time resolution is achieved by using the APV25 front-end readout 
ASIC initially developed for the CMS detector at CERN~\cite{apv25} operating in multi-peak mode. 
The APV25 chips, wire-bonded to the end of the sensor, are mounted on FR4 hybrid boards. A half-module is built by 
a sensor and hybrid glued to a polyimide-laminated carbon fiber composite backing keeping cooling and electrical services 
outside the tracking volume\cite{paper_testrun}. 
For tracking layers 4-6, double length half-modules are built with hybrids at each end to increase 
acceptance. Placing two half-modules, with each hybrid end sandwiched around an aluminum 
cooling block, back-to-back with a 50 or 100~mrad stereo angle gives the required 3D space point resolution and 
conducts heat from the sensor to the cooled support structure to remove $\sim1.7$~W of power per hybrid. The silicon operates at 
-10$^{\circ}$C to withstand high localized radiation doses up to $1 \times 10^{14}$~1~MeV neutron eq. fluence. 
Critical to minimizing the multiple scattering uncertainty, a material budget of less than 0.7\%~$X_{0}$ per layer is obtained. 
Layer 1-3 and 4-6 , top and bottom modules, are mounted directly on cooled U-channels precision mounted on kinematic mounts.  
A lever arm extending upstream to a motor controlled high-precision vertical linear shift gives an adjustable distance to the beam 
plane for the L1-3 the top and bottom halves of the tracker. Mounting the four U-channel support structure on rails allows removal 
of the tracker from the vacuum chamber with minimal intervention. Figure~\ref{fig:svt1} shows the four U-support channels 
complete with modules.

Six analog samples from the APV25 chips at 41.66~MHz from up to four hybrid boards are sent on twisted pair magnet wire 
to a total of 10 Front End Boards (FEB). Each FEB digitizes and transfers up to 3.3~Gb/s of data using high-speed serial links to 
Xilinx Zynq based data processing modules on the ATCA based SLAC RCE platform~\cite{paper_testrun} for zero suppression 
and event building. 
\begin{figure}[!t]
\centering
\includegraphics[width=3.5in]{2014-1111-1596-heavy_photon_search-2.jpg}
\caption{The four U-channels, layers 1-3 (closest) with the detector modules mounted. }
\label{fig:svt1}
\end{figure}
Each FEB also handles power regulation and monitoring as well as high voltage sensor bias distribution to each of the attached 
hybrids. To shorten the distance of analog 
signal transfer, the FEBs are fitted inside the vacuum chamber, pressed against thermal pads on each side of a 1/2$"$ cooled support plate on the upstream positron side, which has a less intense radiation environment. Borated high-density polyethylene is used to 
further lower the risk of damage from radiation emitted by the nearby target.

Multiple scattering of the low momentum electrons are the dominant factor in limiting tracking performance. Impact parameters 
between 350~$\mu$m and 100~$\mu$m for track momentum between 0.25 and 1.7~GeV$/$c and a 5\% momentum resolution 
is achieved. With a precise 3D vertex resolution from the three most upstream layers with large stereo angles, and using the 
$\sim40~\mu$m wide beam spot as a constraint, full track reconstruction simulation show a $\sim 10^7$ rejection of prompt 
backgrounds. This ensures good sensitivity for \Aprime{} decay lengths larger than 1~cm. 

This contribution will discuss the first results from the HPS experiment. The achieved hit time resolution, tracking efficiency  and 
momentum resolution will be presented that will pave the way for the first physics results. 




% An example of a floating figure using the graphicx package.
% Note that \label must occur AFTER (or within) \caption.
% For figures, \caption should occur after the \includegraphics.
% Note that IEEEtran v1.7 and later has special internal code that
% is designed to preserve the operation of \label within \caption
% even when the captionsoff option is in effect. However, because
% of issues like this, it may be the safest practice to put all your
% \label just after \caption rather than within \caption{}.
%
% Reminder: the "draftcls" or "draftclsnofoot", not "draft", class
% option should be used if it is desired that the figures are to be
% displayed while in draft mode.
%
%\begin{figure}[!t]
%\centering
%\includegraphics[width=3.5in]{fulldetector.pdf}
% where an .eps filename suffix will be assumed under latex, 
% and a .pdf suffix will be assumed for pdflatex; or what has been declared
% via \DeclareGraphicsExtensions.
%\caption{Daily abstract submission rate of the 2007 NSS-MIC. }
%\label{fig_sim}
%\end{figure}

% Note that IEEE typically puts floats only at the top, even when this
% results in a large percentage of a column being occupied by floats.


% An example of a double column floating figure using two subfigures.
% (The subfig.sty package must be loaded for this to work.)
% The subfigure \label commands are set within each subfloat command, the
% \label for the overall figure must come after \caption.
% \hfil must be used as a separator to get equal spacing.
% The subfigure.sty package works much the same way, except \subfigure is
% used instead of \subfloat.
%
%\begin{figure*}[!t]
%\centerline{\subfloat[Case I]\includegraphics[width=2.5in]{subfigcase1}%
%\label{fig_first_case}}
%\hfil
%\subfloat[Case II]{\includegraphics[width=2.5in]{subfigcase2}%
%\label{fig_second_case}}}
%\caption{Simulation results}
%\label{fig_sim}
%\end{figure*}
%
% Note that often IEEE papers with subfigures do not employ subfigure
% captions (using the optional argument to \subfloat), but instead will
% reference/describe all of them (a), (b), etc., within the main caption.


% An example of a floating table. Note that, for IEEE style tables, the 
% \caption command should come BEFORE the table. Table text will default to
% \footnotesize as IEEE normally uses this smaller font for tables.
% The \label must come after \caption as always.
%
%\begin{table}[!t]
%% increase table row spacing, adjust to taste
%\renewcommand{\arraystretch}{1.3}
% if using array.sty, it might be a good idea to tweak the value of
% \extrarowheight as needed to properly center the text within the cells
%\caption{An Example of a Table}
%\label{table_example}
%\centering
%% Some packages, such as MDW tools, offer better commands for making tables
%% than the plain LaTeX2e tabular which is used here.
%\begin{tabular}{|c||c|}
%\hline
%One & Two\\
%\hline
%Three & Four\\
%\hline
%\end{tabular}
%\end{table}


% Note that IEEE does not put floats in the very first column - or typically
% anywhere on the first page for that matter. Also, in-text middle ("here")
% positioning is not used. Most IEEE journals use top floats exclusively.
% Note that, LaTeX2e, unlike IEEE journals, places footnotes above bottom
% floats. This can be corrected via the \fnbelowfloat command of the
% stfloats package.



% if have a single appendix:
%\appendix[Proof of the Zonklar Equations]
% or
%\appendix  % for no appendix heading
% do not use \section anymore after \appendix, only \section*
% is possibly needed

% use appendices with more than one appendix
% then use \section to start each appendix
% you must declare a \section before using any
% \subsection or using \label (\appendices by itself
% starts a section numbered zero.)
%
\vfill





\appendices
\section{}
If  I need an appendix it should go here.

% use section* for acknowledgement
\section*{Acknowledgment}
The authors are grateful for the support from Hall~B at JLab and especially the Hall~B engineering 
group for support during installation and decommissioning. They also would like to commend the 
CEBAF personnel for good beam performance, especially the last few hours of operating CEBAF6. 
The tremendous support from home institutions and supporting staff also needs praise from the 
authors. 
%Work supported by the U.S. Department of Energy under contract number DE-AC02-76SF00515, 
%the National Science Foundation,  
%French Centre National de la Recherche Scientifique and 
%Italian Istituto Nazionale di Fisica Nucleare. 
%Rouven Essig is supported in part by the Department of Energy Early Career research program 
%DESC0008061and by a Sloan Foundation Research Fellowship. Authored by Jefferson Science 
%Associates, LLC under under U.S. Department of Energy contract No. DE-AC05-06OR23177.

% references section

% can use a bibliography generated by BibTeX as a .bbl file
% BibTeX documentation can be easily obtained at:
% http://www.ctan.org/tex-archive/biblio/bibtex/contrib/doc/
% The IEEEtran BibTeX style support page is at:
% http://www.michaelshell.org/tex/ieeetran/bibtex/
%\bibliographystyle{IEEEtran}
% argument is your BibTeX string definitions and bibliography database(s)
%\bibliography{IEEEabrv,../bib/paper}
%
% <OR> manually copy in the resultant .bbl file
% set second argument of \begin to the number of references
% (used to reserve space for the reference number labels box)
%\begin{thebibliography}{1}
%
%\bibitem{IEEEhowto:kopka}
%H.~Kopka and P.~W. Daly, \emph{A Guide to \LaTeX}, 3rd~ed.\hskip 1em plus
% 0.5em minus 0.4em\relax Harlow, England: Addison-Wesley, 1999.
%
%\bibitem{IEEEPDFRequirement401}
%IEEE Content Engineering, \emph{PDF Specification for IEEE Xplore}. Available: http://www.ieee.org/portal/cms\_docs/pubs/confstandards/pdfs/IEEE-PDF-SpecV401.pdf.
%
%\end{thebibliography}

\begin{thebibliography}{1}

\bibitem{pamela}
O. Adriani {\it et al.} [PAMELA Collaboration], Nature {\bf 458}, 607 (2009) 
%[arXiv:0810.4995 [astro- ph]],
%\bibitem{pamela2} 
%O. Adriani et al. [PAMELA Collaboration], Phys. Rev. Lett. 106, 201101 (2011) [arXiv:1103.2880 [astro-ph.HE]].
\bibitem{fermi} 
M. Ackermann {\it et al.} [Fermi LAT Collaboration], Phys. Rev. D {\bf 82}, 092004 (2010) 
%[arXiv:1008.3999 [astro-ph.HE]].
%\bibitem{fermi2} 
%M. Ackermann et al. [The Fermi LAT Collaboration], Phys. Rev. Lett. 108, 011103 (2012) [arXiv:1109.0521 [astro-ph.HE]].
%\bibitem{atic} 
%J. Chang et al., Nature 456, 362 (2008).
%\bibitem{hess1} 
%F. Aharonian et al. [H.E.S.S. Collaboration], Phys. Rev. Lett. 101, 261104 (2008) [arXiv:0811.3894 [astro-ph]].
%\bibitem{hess2} 
%F. Aharonian et al. [H.E.S.S. Collaboration], Astron. Astrophys. 508, 561 (2009) [arXiv:0905.0105 [astro-ph.HE]].
\bibitem{nima}
N. Arkani-Hamed, D. P. Finkbeiner, T. R. Slatyer and N. Weiner, Phys. Rev. D {\bf 79}, 015014 (2009).
%M. Pospelov and A. Ritz, 
%�Astrophysical Signatures of Secluded Dark Matter,� 
%Phys. Lett. B 671 (2009) 391 [arXiv:0810.1502 [hep-ph]],
%M. Cirelli, M. Kadastik, M. Raidal and A. Strumia, Nucl. Phys. B 813, 1 (2009) [arXiv:0809.2409 [hep-ph]].,
%I. Cholis, D. P. Finkbeiner, L. Goodenough and N. Weiner, JCAP 0912, 007 (2009) [arXiv:0810.5344 [astro-ph]], I. Cholis, G. Dobler, D. P. Finkbeiner, L. Goodenough and N. Weiner, Phys. Rev. D 80, 123518 (2009) [arXiv:0811.3641 [astro-ph]].
\bibitem{holdom}
B. Holdom, Phys. Lett. B {\bf 166}, 196 (1986),
%\bibitem{galison}
P. Galison {\it et al,} 
%�Two Z�s Or Not Two Z�s?,� 
Phys. Lett. B {\bf 136} (1984) 279
\bibitem{Hewett:2012ns} 
  J.~L.~Hewett, {\it et al.},
  ``Fundamental Physics at the Intensity Frontier,''
  arXiv:1205.2671 [hep-ex].
  %%CITATION = ARXIV:1205.2671;%%
%\bibitem{apex1} 
%R.Essig et al, JHEP1102,009(2011)[arXiv:1001.2557 [hep-ph]].
\bibitem{darkforces}
�Dark2012: Dark Forces at Accelerators�, http://www.lnf.infn.it/conference/dark/index.php
%\bibitem{apex} 
%S. Abrahamyan et al. [APEX Collaboration], Phys. Rev. Lett. 107, 191804 (2011) [arXiv:1108.2750 [hep-ex]].
%\bibitem{mami} 
%H. Merkel et al. [A1 Collaboration], Phys. Rev. Lett. 106, 251802 (2011) [arXiv:1101.4091 [nucl-ex]].
%\bibitem{darklight}  
%M. Freytsis et al, �Dark Force Detection in Low Energy E-P Colli- sions,� JHEP 1001 (2010) 111 [arXiv:0909.2862 [hep-ph]].
\bibitem{bible} J. D. Bjorken, R. Essig, P. Schuster and N. Toro, Phys. Rev. D {\bf 80}, 075018 (2009) [arXiv:0906.0580 [hep-ph]].
%\bibitem{hiddensector1} M. Goodsell and A. Ringwald, �Light hidden-sector U(1)s in string compactifications,� Fortsch. Phys. 58, 716 (2010) [arXiv:1002.1840 [hep-th]].
%\bibitem{hiddensector1}  P. Candelas, G. T. Horowitz, A. Strominger and E. Witten, �Vacuum Configurations for Superstrings,� Nucl. Phys. B 258, 46 (1985).
%\bibitem{hiddensector1}  E. Witten, �New Issues in Manifolds of SU(3) Holonomy,� Nucl. Phys. B 268, 79 (1986).
%\bibitem{hiddensector1} S. Andreas, M. D. Goodsell and A. Ringwald, �Dark matter and Dark Forces from a super-symmetric hidden sector,� arXiv:1109.2869 [hep-ph].
%\bibitem{hiddensector1} J. Jaeckel and A. Ringwald, �The Low-Energy Frontier of Particle Physics,� Ann. Rev. Nucl. Part. Sci. 60, 405 (2010) [arXiv:1002.0329 [hep-ph]].
%\bibitem{mass1} P. Fayet, �U-Boson Production in E+ E- Annihilations, Psi and Upsilon Decays, and Light Dark Matter,� Phys. Rev. D 75 (2007) 115017 [arXiv:hep-ph/0702176].
%\bibitem{mass1} C. Cheung, J. T. Ruderman, L. T. Wang and I. Yavin, �Kinetic Mixing as the Origin of Light Dark Scales,� Phys. Rev. D 80 (2009) 035008 [arXiv:0902.3246 [hep-ph]].
%\bibitem{mass1} N. Arkani-Hamed and N. Weiner, JHEP 0812, 104 (2008) [arXiv:0810.0714 [hep-ph]].
%\bibitem{mass1} D. E. Morrissey, D. Poland and K. M. Zurek, �Abelian Hidden Sectors at a Gev,� JHEP 0907 (2009) 050 [arXiv:0904.2567 [hep-ph]].
%\bibitem{g-2_constraints}
%M.~Pospelov, Phys.\ Rev.\ D {\bf 80}, 095002 (2009) [arXiv:0811.1030];
 %G. W. Bennett et al., [Muon G2 Collaboration], Phys. Rev. D{\bf 73} 072003 (2006)[hep?ex/0602035].
%\bibitem{Endo:2012hp} M.~Endo, K.~Hamaguchi and G.~Mishima, Phys.\ Rev.\ D {\bf 86}, 095029 (2012) [arXiv:1209.2558 [hep-ph]].
%\bibitem{cmb} J. B. Dent, F. Ferrer and L. M. Krauss, 
%�Constraints on Light Hidden Sector Gauge Bosons from Supernova Cooling,� arXiv:1201.2683 [astro-ph.CO].
%\bibitem{fixedtargetexp} J. D. Bjorken et al., Phys. Rev. D 38 (1988) 3375, E. M. Riordan et al., Phys. Rev. Lett. 59 (1987) 755, A. Bross, M. Crisler, S. H. Pordes, J. Volk, S. Errede and J. Wrbanek, Phys. Rev. Lett. 67 (1991) 2942.
%\bibitem{collider}
%KLOE-2 Collaboration, %�Search for a Vector Gauge Boson in Phi Meson Decays with theKLOE Detector,� Phys. Lett. B 706 (2012) 251 [arXiv:1110.0411 [hep-ex]], 
%M. Reece and L. T. Wang, 
%�Searching for the Light Dark Gauge Boson in Gev-Scale Experi-ments,� JHEP 0907 (2009) 051 [arXiv:0904.1743 [hep-ph]],
%B. Aubert et al. [BABAR Collaboration], Phys. Rev. Lett. 103, 081803 (2009) [arXiv:0905.4539 [hep-ex]].
\bibitem{proposal_full}
A. Grillo {\it et al.} [HPS Collaboration], JLab PAC37 PR-11-006, http://www.jlab.org/exp prog/PACpage/PAC37/proposals/Proposals/
\bibitem{d0run2b}
D. S. Denisov and S. Soldner-Rembold, FERMILAB-PROPOSAL-0925.
\bibitem{apv25}
M.J. French {\it et al.}, Nucl. Instr. Meth. A, {\bf 466}, 359-365 (2001)
\bibitem{paper_testrun}
P. Hansson Adrian {\it et al.} [HPS Collaboration], The Heavy Photon Search test detector, Nucl. Instr. Meth. A 777 (2015), 91-101, http://dx.doi.org/10.1016/j.nima.2014.12.017
%M. Battaglieri, S. Boyarinov, S. Bueltmann, V. Burkert, A. Celentano, G. Charles, W. Cooper, C. Cuevas, N. Dashyan, R. DeVita, C. Desnault, A. Deur, H. Egiyan, L. Elouadrhiri, R. Essig, V. Fadeyev, C. Field, A. Freyberger, Y. Gershtein, N. Gevorgyan, F.-X. Girod, N. Graf, M. Graham, K. Griffioen, A. Grillo, M. Guidal, G. Haller, P. Hansson Adrian, R. Herbst, M. Holtrop, J. Jaros, S. Kaneta, M. Khandaker, A. Kubarovsky, V. Kubarovsky, T. Maruyama, J. McCormick, K. Moffeit, O. Moreno, H. Neal, T. Nelson, S. Niccolai, A. Odian, M. Oriunno, R. Paremuzyan, R. Partridge, S.K. Phillips, E. Rauly, B. Raydo, J. Reichert, E. Rindel, P. Rosier, C. Salgado, P. Schuster, Y. Sharabian, D. Sokhan, S. Stepanyan, N. Toro, S. Uemura, M. Ungaro, H. Voskanyan, D. Walz, L.B. Weinstein, B. Wojtsekhowski, The Heavy Photon Search test detector, Nuclear Instruments and Methods in Physics Research Section A: Accelerators, Spectrometers, Detectors and Associated Equipment, Volume 777, 21 March 2015, Pages 91-101, ISSN 0168-9002, http://dx.doi.org/10.1016/j.nima.2014.12.017

%\bibitem{proposal_testrun}
%P. Hansson Adrian {\it et al.} [HPS Collaboration], HPS Test Run Proposal and PAC39 Update, https://confluence.slac.stanford.edu/display/hpsg/Project+Overview,
%\bibitem{pac39}
%P. Hansson et al. [HPS Collaboration], HPS Update PAC 39, https://confluence.slac.stanford.edu/display/hpsg/Project+Overview
%\bibitem{egs5}
%H. Hirayama, Y. Namito, A.F. Bielajew, S.J. Wilderman and W.R. Nelson, SLAC-R-730 (2005) and KEK Report 2005-8 (2005).


\end{thebibliography}





% that's all folks
\end{document}


